%! Author = Patrik Skaloš
%! login = xskalo01

% Preamble
\documentclass[a4paper, 11pt]{article}

% Packages
\usepackage[utf8]{inputenc}
\usepackage[slovak]{babel}
\usepackage{times}
\usepackage[left = 2cm, top = 3cm, total={17cm, 24cm}]{geometry}
\usepackage{hyperref}
\usepackage{url}
\urlstyle{same}

% Document
\begin{document}
  \begin{titlepage}
    \begin{center}
      \Huge
      \textsc{Vysoké učení technické v~Brně} \\
      \huge\textsc{Fakulta informačních technologií} \\
      \vspace{\stretch{0.382}}
      \LARGE
      Typografia a publikovanie\,--\,4. projekt\\
      \Huge
      Citácie\\
      \vspace{\stretch{0.618}}
      {\Large \today\hfill
      Patrik Skaloš}
    \end{center}
  \end{titlepage}


  \section{Úvod}
  Umelá inteligencia (\emph{angl. artificial intelligence}, ďalej \emph{AI}) 
  je dnes na každom rohu a~väčšina z~nás je na nej závislá, pričom o~tom
  ani nemusíme vedieť, keďže je skrytá za vrstvou nazývanou 
  \emph{uživateľské rozhranie}. \\
  \emph{Najčastejšie sa s~ňou človek stretne pri 
  cielených reklamách, vyhľadávaní na internete, na sociálnych sieťach a~
  podobne.} \cite{Markowitz:2020:TakeALookAround}

  \emph{S~názvom \uv{artificial intelligence} prišiel v~roku 1955 John 
  McCarthy, keď tak nazval letnú školu, ktorú organizoval. Práve od toho 
  momentu sa začal pojem AI používať bežne.}
  \cite{Wooldridge:2021:ABriefHistoryOfArtificialIntelligence} 

  \section{Neurónové siete a ich optimalizácia}
  AI je založená na neurónových sieťach. Tie sú, ako to Daniel Shiffman 
  vysvetľuje vo svojej knihe, \emph{výpočtové modely založené na mozgu určené 
  na riešenie istých problémov}. 
  \cite{Shiffman:2012:TheNatureOfCode}

  \subsection{Problémy neurónových sietí}
  \emph{Neurónové siete majú svoje nevýhody, ako napríklad tzv. problém vedra 
  bez dna}, ktorý opisuje Erik Larson vo svojej knihe The Myth of Artificial 
  Intelligence \cite{Larson:2021:TheMythOfArtificialIntelligence}. 

  Tento problém hovorí o~tom, že \emph{(zatiaľ) nepoznáme spôsob, ako vyriešiť 
  jednoduchý problém bez toho, aby sme spracovali obrovské množstvo 
  bezvýznamných informácií (na rozdiel od ľudského mozgu, ktorý dokáže 
  efektívne vyberať len informácie, ktoré predpokladá, že sú dôležité)}. 

  Je preto zvyčajne nevyhnutné filtrovať informácie, ktoré umelej inteligencii
  poskytneme. Na zvyšovanie efektivity však poznáme aj množstvo ďalších metód. 
  Napríklad metóda \emph{dropout} hovorí o~\emph{náhodnom mazaní neurónov 
  počas evolúcie, ktoré dokáže zvýšiť výkon pri učení pod dohľadom}. 
  \cite{Srivastava:2014:Dropout} 
  Metóda z~názvom \emph{normalizácia várok údajne dokáže výrazne zvýšiť 
  rýchlosť učenia, zjednodušiť inicializáciu a~niekedy dokonca nahradiť 
  funkciu metódy dropout}. \cite{Ioffe:2015:BatchNormalization}

  \subsection{Optimalizácia neurónových sietí}
  Optimalizovať algoritmy učenia je dôležitejšie, ako sa na prvý pohľad môže 
  zdať. Neurónová sieť nikdy nebude perfektná, a~každý kúsok optimalizácie
  môže znamenať rýchlejšie dosiahnutie lepšieho výsledku. Na lepších výsledkoch
  záleží najmä v~odvetviach, kde ide o~život. Efektívnejšie učenie má však 
  ďalšie výhody. Napríklad na vývin autonómneho vozidla potrebujeme obrovské 
  množstvo dát z~ciest. Tie sú však obmedzené a~preto sa predpokladá, že 
  vyvinutý systém nie je ani zďaleka dokonalý, \emph{čo má za následok potrebu 
  zdĺhavého testovania, a~práve to vývoj značne spomaľuje}.
  \cite{Tinakova:2020:AutonomneVozidla}
  Ak by sme však dokázali dostatočne zefektívniť učenie nad rovnakým objemom 
  dát, na cestách by sme už dnes videli mnoho autonómnych áut.

  \section{Príklad použitia - obchodovanie na trhu}
  Osobne ma naposledy zaujal koncept predpovede trhu pomocou umelej 
  inteligencie. \emph{Je jasné, že každý trh je ovplyvňovaný neuveriteľným 
  množstvom faktorov a~o~to náročnejšia je analýza väčšieho množstva 
  trhov.}\cite{WU:2017:PerformanceComparison} 

  To je len jeden z~mnohých problémov pri implementácii systému na predpoveď 
  ceny trhu. Kvôli týmto problémom nemôže byť žiaden mechanický systém 
  predpovede dlhodobo ziskový pri obchodovaní na trhoch. 
  Mnohí sa však pokúšajú týmto problémom čeliť pomocou neurónových sietí.

  Petr Volf vo svojej diplomovej práci popisuje, ako dokázal predpovedať trh
  so správnosťou až $95.45\%$.
  \cite{VOLF:2015:VyuzitiNaKomoditnich}

  V~svojej diplomovej práci Martin Kuna popisuje správnosť predpovede trhu
  $74\%$, zaujímavejšie je však tvrdenie, že z~$1.7\times10^{13}$ 
  možných kombinácii dokázala jeho aplikácia vybrať \emph{optimálne portfólio}.
  \cite{Kuna:2010:VyuzitiNaKapitalovych}

  \bibliography{proj4}
  \bibliographystyle{czechiso}

\end{document}
