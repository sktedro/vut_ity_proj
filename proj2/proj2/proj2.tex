%! Author = Patrik Skaloš
%! xlogin = xskalo01

% Preamble
\documentclass[a4paper, 11pt, twocolumn]{article}

% Packages
\usepackage[utf8]{inputenc}
\usepackage[IL2]{fontenc}
\usepackage[czech]{babel}
\usepackage{amsmath, amssymb, amsthm}
\usepackage[left = 1.5cm, top = 2.5cm, total={18cm, 25cm}]{geometry}
\usepackage{color}
\usepackage[unicode]{hyperref}
\usepackage{times}


\newtheorem{definice}{Definice}
\newtheorem{veta}{Věta}

% Document
\begin{document}
    \begin{titlepage}
        \begin{center}
            \Huge
            \textsc{Fakulta informačních technologií} \\
            \textsc{Vysoké učení technické v~Brně} \\
            \vspace{\stretch{0.382}}
            \LARGE
            Typografie a publikování\,--\,2. projekt \\
            Sazba dokumentů a matematických výrazů \\
            \vspace{\stretch{0.618}}
        \end{center}
        {\Large 2021\hfill
        Patrik Skaloš (xskalo01)}
    \end{titlepage}

    \section*{Úvod}
    \label{page2}
    V~této úloze si vyzkoušíme sazbu titulní strany, matematic\-kých
    vzorců, prostředí a dalších textových struktur obvyklých
    pro technicky zaměřené texty (například rovnice (\ref{eq1})
    nebo Definice \ref{def1} na straně \pageref{page2}). Rovněž si vyzkoušíme
    používání odkazů \verb|\ref| a \verb|\pageref|.

    Na titulní straně je využito sázení nadpisu podle optického středu
    s~využitím zlatého řezu. Tento postup byl probírán na přednášce. Dále je
    použito odřádkování se zadanou relativní velikostí 0.4\,em a 0.3\,em.

    V~případě, že budete potřebovat vyjádřit matematickou konstrukci nebo
    symbol a nebude se Vám dařit jej nalézt v~samotném \LaTeX{u}, doporučuji
    prostudovat možnosti balíku maker \AmS-\LaTeX.

    \section{Matematický text}
    Nejprve se podíváme na sázení matematických symbolů a~výrazů v~plynulém
    textu včetně sazby definic a vět s~využitím balíku \texttt{amsthm}. Rovněž
    použijeme poznámku pod čarou s~použitím příkazu \verb|\footnote|. Někdy je
    vhodné použít konstrukci \verb|\mbox{}|, která říká, že text nemá být zalomen.

    \begin{definice}
        \label{def1}
        \emph{Rozšířený zásobníkový automat} (RZA) je definován jako
        sedmice tvaru $A=(Q, \Sigma, \Gamma, \delta, q_0, Z_0, F)$, kde:
    \end{definice}

    \renewcommand{\labelitemi}{$\bullet$}
    \begin{itemize}
        \item $Q$ \emph{je konečná množina} vnitřních (řídicích) stavů,
        \item $\Sigma$ \emph{je konečná} vstupní abeceda,
        \item $\Gamma$ \emph{je konečná} zásobníková abeceda,
        \item $\delta$ \emph{je} přechodová funkce
        $Q \times (\Sigma\cup\{\epsilon\}) \times \Gamma^{\ast}$
        $\rightarrow 2^{Q\times\Gamma^{\ast}}$,
        \item $q_0 \in Q$ \emph{je} počáteční stav, $Z_0 \in \Gamma$ \emph{je} startovací
        symbol zásobníku \emph{a} $F \subseteq Q$ \emph{je množina} koncových stavů.
    \end{itemize}

    Nechť $P=(Q, \Sigma, \Gamma, \delta, q_0, Z_0, F)$ je rozšířený zásobníkový
    automat. \emph{Konfigurací} nazveme trojici
    $(q, w, \alpha) \in Q \times \Sigma^{\ast} \times \Gamma^\ast$, kde $q$ je aktuální stav
    vnitřního řízení, $w$ je dosud nezpracovaná část vstupního řetězce a
    $\alpha = Z_{i_1}Z_{i_2}\dots Z_{i_k}$ je obsah
    zásobníku\footnote[1]{$Z_{i_1}$ je vrchol zásobníku}.


    \subsection{Podsekce obsahující větu a odkaz}
    \begin{definice}
        \label{def2}
        \emph{Řetězec $w$ nad abecedou $\Sigma$ je přijat RZA}
        A~jestliže $(q_0, w, Z_0) \overset{*}{\underset{A}\vdash}
        (q_F, \epsilon, \gamma)$ pro nějaké $\gamma \in \Gamma^{\ast}$ a
        $q_F\!\in\!F$. Množinu $L(A) = \{w \mid w \text{ je přijat RZA A} \} \subseteq
        \Sigma^{\ast} \text{ nazýváme} \emph{ jazyk přijímaný RZA } A$.
    \end{definice}

    Nyní si vyzkoušíme sazbu vět a důkazů opět s~použitím balíku \texttt{amsthm}.
    \begin{veta}
        Třída jazyků, které jsou přijímány ZA, odpovídá \emph{bezkontextovým jazykům}.
    \end{veta}
    \begin{proof}
        V~důkaze vyjdeme z~Definice \ref{def1} a \ref{def2}.
    \end{proof}

    \section{Rovnice a odkazy}
    Složitější matematické formulace sázíme mimo plynulý text. Lze umístit
    několik výrazů na jeden řádek, ale pak je třeba tyto vhodně oddělit,
    například příkazem \verb|\quad|.
    \bigskip
    \begin{equation}
        \sqrt[i]{x^3_i} \quad \text{kde } x_i \text{ je } i
        \text{-té sudé číslo splňující} \quad
        x_i^{x_i^{i^2}+2} \leq y_i^{x_i^4}
        \nonumber
    \end{equation}

    V~rovnici (\ref{eq1}) jsou využity tři typy závorek s~různou explicitně definovanou
    velikostí.
    \bigskip
    \begin{eqnarray}
        \label{eq1}
        x & = & \bigg[\Big\{\big[a+b\big]*c\Big\}^d \oplus 2 \bigg]^{3/2} \\
        y & = & \lim_{x\to\infty} \frac{\frac{1}{\log_{10} x}}{\sin^2 x + \cos^2 x}
        \nonumber
    \end{eqnarray}

    V~této větě vidíme, jak vypadá implicitní vysázení limity
    $\lim_{n\to\infty}f(n)$ v~normálním odstavci textu. Podobně je to i
    s~dalšími symboly jako $\prod^n_{i=1}2^i$ či $\bigcap_{A\in\mathcal{B}} A$. V~případě
    vzorců $\lim\limits_{n\to\infty}f(n)$ a $\prod\limits^n_{i=1}2^i$ jsme si
    vynutili méně úspornou sazbu příkazem \verb|\limits|.
    \bigskip
    \begin{equation}
        \int^a_b g(x)\mathrm{\,d}x\hspace{0.7em} =
        \hspace{0.7em} - \int\limits^b_a f(x)\mathrm{\,d}x
    \end{equation}
    \section{Matice}
    Pro sázení matic se velmi často používá prostředí \texttt{array} a závorky
    (\verb|\left|, \verb|\right|).

    \begin{equation}
        \nonumber
        \left(
        \begin{array}{ccc}
            a-b & \widehat{\xi + \omega} & \pi \\
            \vec{\mathbf{a}} & \overleftrightarrow{AC} & \hat{\beta}
        \end{array}
        \right)
        = 1 \Longleftrightarrow \mathcal{Q} = \mathbb{R}
    \end{equation}

    \begin{equation}
        \nonumber
        \textbf{A} = \left\|
        \begin{array}{cccc}
            a_{11} & a_{12} & \ldots & a_{1n} \\
            a_{21} & a_{22} & \ldots & a_{2n} \\
            \vdots & \vdots & \ddots & \vdots \\
            a_{m1} & a_{m2} & \ldots & a_{mn}
        \end{array}
        \right\| = \left|
        \begin{array}{rl}
            t & u \\
            v & w
        \end{array}
        \right| = tw{-}uv
    \end{equation}

    Prostředí \texttt{array} lze úspěšně využít i jinde.

    \begin{displaymath}
        \binom{n}{k} = \left\{
        \begin{array}{cl}
            0 & \text{pro } k\;< 0 \text{ nebo } k\;> n \\
            \frac{n!}{k!(n-k)!} & \text{pro } 0 \leq k\;\leq n.
        \end{array} \right.
    \end{displaymath}
\end{document}
